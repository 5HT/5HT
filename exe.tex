% copyright (c) 2013 Synrc Research Center

\documentclass[11pt,oneside]{article}
\input{synrc.tex}
\begin{document}

\thispagestyle{empty}
\begin{center}

\begin{minipage}[t]{2cm}
    \includegraphics[scale=0.4]{img/S}
\end{minipage}
\begin{minipage}[t]{12cm}
    \begin{flushright}
        \textsc{{\Large {\bf {\color{Blue}syn}{\color{OrangeRed}rc} research center s.r.o.}}}\\
        \textsc{Roháčova 141/18, Praha 3 13000, Czech Republic}\\
    \end{flushright}
\end{minipage}

\vspace{3cm}

    \vspace{3cm}   {\Large \bf Формальні методи верифікованих середовищ \\послідовних обчислень узгоджених та розподілених \\паралельних систем\par}
    \vspace{3cm}   {\Large Максим Сохацький\par}
    \vspace{6cm}   {\Large Київський Політехнічний Інститут\\}
    \vspace{0.3cm} {\Large Листопад 2015}
\end{center}

\vspace{2cm}
\newpage
\section{Глава 1. Вступ}

\vspace{1cm}

\subsection{Що означає назва}

   {\bf Формальні методи верифікованих середовищ
   послідовних обчислень розподілених узгоджених паралельних систем.}

\vspace{1cm}

   \paragraph{}
   Формальні методи тут розглядаютьяся у широкому сенсі, особливо нас цікавитиме
   та частина математичного забезпечення яке можно формалізувати для доведення
   машинним способоб, що унеможливлює виникнинення цілого класу помилок.
   Зокрема розглядатимуться усі теорії які застосовують до подібних систем
   з підвищеними запитами якості.

   \paragraph{}
   Розглядаючи такі системи, які піддаються формальній верифікації, кажуть про
   сертифіковане або верифіковане програмне забезпечення. А у випадку коли такі
   системи розповсюджуються на всі прошарки моделі OSI, то кажуть про замкнені
   верифіковані середовища. Саме розробка такого середовища є предметом даної роботи.

   \paragraph{}
   Чому саме послідовні обчислення? У цій частині мова йде про природу обчислень,
   чи природу послідовності висловлювань у формальній логіці. Саме лінійність кінцевого
   процессу виконання певного структурованого або циклічного алгоритму відіграє ключову
   роль у моделюванні віртуальних обчислювальних середовищ. Такі лінеаризовані системи
   уже показали свою ефективність в певному класі обчислень, таких як ланцюгова реплікація,
   реактивні системи, та інші моделі напівгруп навколо певних типів -- протоколів взаємодії між процесами.
   Крім того такі послідовності подій піддаються статистичній обробці для визначення первних кластеризацій
   та інших кореляцій у просторі та часі, тому можна говорити про певні зручні, нормалізовані системи типів для
   такого роду маніпуляцій.

   \paragraph{}
   Побудова розподілених та паралельних, тобто здатних виконуватися на багатьох машинах одночасно, та
   узгоджених, тобто не блокуючих, а значить лінеаризованих систем управліннями процессами є кінцевим
   результатом який очікується від цієї роботи.

\newpage
\subsection{Контекст та сфера дослідження}

\vspace{1cm}
   За багато років кількість теорій, які використовуються для побудови програмного забезпечення значно розширилися:
   починаючи з теорії компіляції сучаних функціональних мов, та систем програмування на основі теорії типів,
   включаючи сучасні моделі обчислень, які побудовані на основі лямбда числення та числення процесів, закінчуючи віртуальними
   машинами які працююсь у семантиці захищених, простих за структурою процесів, час яких розподіляється
   у прозорий та ефективний спосіб.

   \paragraph{}
   Сучасний розвиток техніки та теоретична межа швидкості обробки процесорів вивів на передній план алгоритми та структури
   данних які ефективно використоують розподілені у просторі та часі ресурси, як то об’єми памяті та обчислювальні потужності.
   Принципи та підходи паралельного та узгодженого програмування дають змогу масштабувати системи та обчисленя, однак
   анонсують нові теорії для забезпечення коректності в умавах підвищеної складності алгоритмів у розподілених системах,
   таких як алгоритми забезпечення консистентності та транзакційності у розподілених системах PAXOS та CR.

   \paragraph{}
   З розвитком систем програмування, підвищилась якість інструментів для автоматичного доведення коректності,
   починаючи від інструментів побудованих на темпоральній логіці як TLA+ та EventML закінчуючи системами
   побудованими на індуктивних типах і які вже використовуються метематиками як Agda та Coq.

   \paragraph{}
   Багато методологій виникло і багато підходів були випробувані в польових умовах на різних, по критичності, рівнях якості.
   Починаючи від асемблерів, через процедурний та об’єктно-орієнтований підхід, до функціональних мов програмування
   з розвиненими системами типів, таких як Haskell та ML, а також мов, спеціалізованих для розподілених систем як Erlang.

   \paragraph{}
   Однак побудова замкненої екосистеми

\newpage
\subsection{Об’єкт дослідження}
\vspace{1cm}
   Об’єктом дослідження є усі можливі моделі обчислювальних середовищ в основному придатних
   до верифікованого аналізу та обробки формальними методами. Самі формальні методи теж є
   частиною об’єкта дослідження. Ми досліджуємо ті структури та алгоритми які дадуть
   максимально ефективний спосіб кодування та виконнання, забезпечуючи при цьому семантику, яка використовуються
   для машинного доведення коректності роботи алгоритмів та непротиречивості структур даних.

\newpage
\subsection{Теоретичні сутності}
\vspace{1cm}
   У таксономії теоритичних сутностей умовно визначатимемо
   графічні топологічно- структурні, аналітичні, статистичні та логічні типи теорій, які
   ми використовуватимемо для опису комплесної теорії верифікованих середовищ
   послідовних обчислень розподілених узгоджених паралельних систем.\\

   {\bf Теорія масового обслуговування} \\ \\
   Теорія массового обслуговування застосовується для побудови
   статистичних моделей та запобігання відмов. Перші роботи у цій області
   належать шведському математику Агнеру Крурупу Ерлангу, який займався
   дослідженнями трафіка у телефонних мережах. Модель масового обслуговування достатньо
   адекватно описує роботу віртуальної машини Erlang, де клієнти -- це процеси,
   які мають черги повідомлень, та здатні відправляти заявки на обслуговування
   у такі самі черги інших процесів. Ці заявки, чи повідомлення складають певний
   протокол взаємодії у системі таких процесів. Тому тут теорія масового обслуговування
   застосовується для визначення пропускної здатності системи.\\

   {\bf Пі-числення}\\ \\
   Теорія Пі-числення Роберта Мілнера є основним формалізмом обчислювальної
   теорії та її імплементації. З часів виникнення CSP числення розробленого Хоаром,
   Мілнеру вдалося значно розширити та адаптувати теорію до сучасних
   телекомунікаційних вимог, як наприклад хендовери в мобільних мережах.
   Основні теорми в моделі Пі-числення стосуються непротиречивості та неблокованості
   у синхронному виконанні мобільних процесів. Так як сучасний Web можно розглядати
   як телекомунікаційну систему, тому у розробці додатків можна покладатися у тому
   числі і на такі моделі як Пі-числення.\\

   {\bf Мережі Петрі} \\ \\
   Мережі Петрі в даній роботі використовуються як прототип графічного
   лінгвістичного засобу структури категорій та системи типів. Оскільки
   такі графічні засоби як UML та різноманітні окремі технологічні
   стандарти як то BPMN більше допомагають ніж мішають, була розроблена
   також і графічна мова на базі графічної мови мереж Петрі, оскільки їх
   семантика ділить один простір з тематикою даної роботи. Ми візьмемо
   лінгвістичне забезпечення Мереж Петрі як прототип для нашої власної
   мови візуалізації структур обчислень.\\

\newpage
   {\bf Теорія категорій} \\ \\
   Теорія категорій як робоча структурна теорія системи типів мови Exe та
   категоріальна семантика числення процесів. Можна було би використовувати абстракну алгебру загалом,
   та теорія напівгруп, а саме напівгруп активностей, проте ми будемо використовувати
   категоріальну семантику, що стало можливо завдяки роботам Лавіра та іншим.\\

   {\bf Лямбда числення} \\ \\
   Лямбда-числення як основна абстракція обчислювальної віртуальної машини.
   Будучи внутрішньою мовою декартово-замкненої категорії лямбда числення окрім змінних
   та констант у вигляді термів пропонує операції абстракції та аплікації, що визначає
   достатньо лаконічну та потужну структуру обчислень з функціями вищих порядків,
   та метатпізаціями, такими як System F, яка була запропонована
   вперше Робіном Мілнером в мові ML, та зараз присутня в більш складних,
   таких як System F$\omega$ системах. \\

   {\bf Темпоральна логіка} \\ \\
   Темпоральна логіка як індуктивна теорія верифікації розподілених алгоритмів
   застосовується до доведення коррекності усіх нормалізованих підсистем. На основі
   теорії \cite{tla+} Леслі Лампорта, за яку він отримав премію Тюрінга. \\

   {\bf Індуктивні типи} \\ \\
   Системи з залежними типами як верифікаційні математичні формальні моделі
   для доведення корректності.

\newpage
\subsection{Предмет дослідження}
\vspace{1cm}
   Категоріальний погляд на теорію типів ізоморфних до системи типів числення процесів.
   Визначення характеристик нормалізованих структур данних для верифікованих та розподілених обчислень.
   Побудова системи типів метамоделі Exe яка відповідає даним характеристикам.
   Повний детальний аналіз OSI стека.
   Розробка системи додатків які покривають верхні частини моделі OSI та здатні виконуватися без операційної системи.

\subsection{Задача дослідження}
\vspace{1cm}
   Реалізація усіх верхніх компонентів OSI на базі метамоделі Exe на мові Erlang.
   Побудова зручної сучасної гнучкої верифікованої теорії
   з компактною системою типів є основною задачою даного дослідження.

\subsection{Мета дослідження}
\vspace{1cm}
   Метою дослідження є збезпечення виконання критеріїв відповідності
   показників ефективності, детермінованості та якості шляхом впровадження
   результату даної теорії, комплексу програмного забезпечення у виробничий процес.

\newpage
\begin{thebibliography}{9}

\bibitem{tla+}   Leslie Lamport. \textit{Specifying Systems} 2004.
\bibitem{erl}    Joe Armstrong. \textit{Making reliable distributed systems in the presence of sodware errors} 2003.
\bibitem{commpi} Robin Milner. \textit{Communicating and Mobile Systems: The $\pi$-calculus.} 1999.
\bibitem{polypi} Robin Milner. \textit{The Polyadic $\pi$-Calculus: A Tutorial.} 1993.
\bibitem{mass}   Коваленко. \textit{Теория Массового Обслуживания.} 1965.
\end{thebibliography}

\vspace{3\baselineskip}
\paragraph{}
З повагою, команда Synrc.
\paragraph{}
\vspace{3\baselineskip}
\begin{tabular}{ll}
        & Сохацький Максим, технічний директор \\
        & Кирилов Володимир, співзасновник
\end{tabular}


\end{document}
